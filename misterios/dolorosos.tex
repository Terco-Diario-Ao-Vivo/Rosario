\section{Mistérios Dolorosos}
\misterio{AGONIA DE JESUS NO HORTO DAS OLIVEIRAS}
\begin{center}
  Fruto: A detestação dos pecados.
\end{center}

\sayy{Retirou-se Jesus com apóstolos para um lugar chamado Getsêmani e disse-lhes: \say{Assentai-vos aqui, enquanto eu vou ali orar}. E, tomando consigo Pedro e os dois filhos de Zebedeu, começou a se entristecer e a se angustiar. Disse-lhes, então: \say{Minha alma está triste até a morte. Ficai aqui e vigiai comigo}. Adiantou-se um pouco e, prostrando-se com a face por terra, assim rezou: \say{Meu Pai, se é possível, afasta de mim este cálice! Todavia não se faça o que eu quero, mas sim o que tu queres.}}

\begin{flushright}
  --- (\textit{cf.} Mt 26:36-39)
\end{flushright}

\begin{center}
  \begin{tikzpicture}[
  scale=0.8,
  every node/.style={anchor=west, align=justify, text width=14cm, font=\fontsize{10}{12}\selectfont},
]

\linha

\conta{para honrar os divinos retiros que fez Jesus em Sua vida, principalmente no horto...}

\conta{para honrar suas orações humildes e fervorosas durante a vida e na véspera da Paixão...}

\conta{para honrar a paciência e doçura com que suportou Seus Apóstolos, particularmente no Jardim do Getsêmani...}

\conta{para honrar o tédio de Sua Alma durante toda a Sua vida, principalmente no Jardim do Getsêmani...}

\conta{para honrar os rios de sangue que a dor fez brotar de Seu ser adorável...}

\conta{para honrar o consolo que teve por bem aceitar de um Anjo na agonia...}

\conta{para honrar Sua conformidade com a Vontade do Pai, apesar da repugnância ao sofrimento sentida pela Sua natureza humana...}

\conta{para honrar a ignomínia que suportou causada pela traição do Apóstolo Judas Iscariotes e Sua consequente prisão pelos judeus...}

\conta{para honrar a coragem com que saiu ao encontro dos Seus algozes e a força da palavra com que os lançou por terra e os levantou...}

\conta{para honrar a humilhação que suportou e a tristeza que sentiu ao ser abandonado por Seus Apóstolos...}

\jaculatorias
\end{tikzpicture}
\end{center}
\newpage


\misterio{FLAGELAÇÃO DE JESUS}

\begin{center}
  Fruto: A mortificação da carne.
\end{center}

\sayy{Era costume que o governador soltasse um preso a pedido do povo em cada festa de Páscoa. Ora, havia naquela ocasião um prisioneiro famoso, chamado Barrabás. Pilatos dirigiu-se ao povo reunido: Qual quereis que eu vos solte: Barrabás ou Jesus, que se chama Cristo? Mas os príncipes dos sacerdotes e os anciãos persuadiram o povo que pedisse a libertação de Barrabás e fizesse morrer Jesus. O governador tomou então a palavra: Qual dos dois quereis que eu vos solte? Responderam: Barrabás! Pilatos perguntou: Que farei então de Jesus? Todos responderam: Seja crucificado! O governador tornou a perguntar: Mas que mal fez ele? E gritavam ainda mais forte: Seja crucificado! Libertou então Barrabás, mandou açoitar Jesus e lho entregou para ser crucificado.}

\begin{flushright}
  --- (\textit{cf.} Mt 27:15-26)
\end{flushright}

\begin{center}
  \begin{tikzpicture}[
  scale=0.8,
  every node/.style={anchor=west, align=justify, text width=14cm, font=\fontsize{10}{12}\selectfont},
  ]
  \linha

\conta{Honrar o sofrimento que Jesus sentiu causado pelas cordas com que foi atado}
\conta{Honrar a dor e vergonha que Jesus suportou por causa da bofetada que recebeu na casa de Caifás}
\conta{Honrar a tristeza que sentiu por causa das negações de Pedro na casa de Caifás}
\conta{Honrar as ignomínias que aguentou calado na casa de Herodes, quando Lhe puseram veste branca}
\conta{Honrar o opróbrio que suportou no despojamento de Suas vestes}
\conta{Honrar os desprezos e insultos que sofreu de Seus verdugos pela Sua nudez}
\conta{Honrar o sofrimento que aguentou por causa das varas espinhosas e os açoites cruéis com que foi golpeado}
\conta{Honrar a humilhação que Lhe foi causado na coluna em que foi atado}
\conta{Honrar o sangue que derramou e as chagas que recebeu}
\conta{Honrar a Sua queda devido à fraqueza pelo sangue que derramou}

  \jaculatorias
  \end{tikzpicture}

\end{center}
\newpage

\misterio{COROAÇÃO DE ESPINHOS}
\begin{center}
  Fruto: A mortificação do orgulho.
\end{center}

\sayy{Os soldados do governador conduziram Jesus para o pretório e o rodearam com todo o pelotão. Arrancaram-lhe as vestes e lhe colocaram um manto escarlate. Depois, trançaram uma coroa de espinhos, meteram-lha na cabeça e lhe puseram na mão uma vara. Os soldados ajoelharam-se diante dele e diziam com escárnio: Salve, rei dos judeus! Cuspiam-lhe no rosto e, tomando da vara, davam-lhe golpes na cabeça.}

\begin{flushright}
  --- (\textit{cf.} Mt 27:27-31)
\end{flushright}

\begin{center}
  \begin{tikzpicture}[
  scale=0.8,
  every node/.style={anchor=west, align=justify, text width=14cm, font=\fontsize{10}{12}\selectfont},
  ]
  \linha

\conta{Honrar o opróbrio que suportou no despojamento de Suas vestes pela terceira vez}
\conta{Honrar a Sua coroação de espinhos}
\conta{Honrar a mansidão com que aceitou o véu com que Lhe vendaram os olhos}
\conta{Honrar a paciência que aguentou as dores, pelas bofetadas, e as humilhações, pelos escarros com que Lhe cobriram o rosto}
\conta{Honrar a força que teve para suportar o andrajo que Lhe puseram sobre os ombros}
\conta{Honrar ter se deixado pôr uma cana em Suas mãos taumaturgas}
\conta{Honrar a dor que sofreu causada pela pedra pontiaguda sobre a qual O sentaram}
\conta{Honrar a docilidade que demonstrou aos que Lhe desferiram ultrajes e os insultos}
\conta{Honrar o sangue e os suores que saíam de Sua cabeça adorável}
\conta{Honrar a grande dor que suportou ao ter os cabelos e a barba arrancados}

  \jaculatorias
  \end{tikzpicture}

\end{center}
\newpage

\misterio{JESUS CARREGA A CRUZ EM DIREÇÃO AO CALVÁRIO.}
\begin{center}
  Fruto: Paciência nas tribulações.
\end{center}

\sayy{Depois de terem escarnecido dele, lhe tiraram a púrpura, deram-lhe de novo as vestes e o conduziram para fora para que fosse crucificado. Passava por ali certo homem de Cirene, chamado Simão, que vinha do campo e o obrigaram a levar a cruz. Depois conduziram Jesus ao lugar chamado Gólgota, que quer dizer lugar do crânio.}


\begin{flushright}
  --- (\textit{cf.} Mt 27:31-33)
\end{flushright}

\begin{center}
  \begin{tikzpicture}[
  scale=0.8,
  every node/.style={anchor=west, align=justify, text width=14cm, font=\fontsize{10}{12}\selectfont},
  ]
  \linha

\conta{Honrar a apresentação de Nosso Senhor diante do povo com o \sayy{Ecce Homo}}
\conta{Honrar a decepção que sentiu na rejeição que sofreu pelo Seu próprio povo que escolheu Barrabás}
\conta{Honrar como se calou contra os falsos testemunhos que contra Ele deram}
\conta{Honrar a Sua aceitação da condenação à morte}
\conta{Honrar o amor com que abraçou e beijou a Cruz}
\conta{Honrar o trabalho espantoso que teve em carregá-la}
\conta{Honrar a perseverança que continuou o desígnio do Pai apesar das quedas de pura debilidade do corpo sob o peso da Cruz}
\conta{Honrar o encontro doloroso com Sua Santa Mãe}
\conta{Honrar o sudário colocado sobre Sua Santa Face}
\conta{Honrar Suas lágrimas, as de Sua Santa Mãe e das piedosas mulheres que O seguiram até o Monte Calvário}

  \jaculatorias
  \end{tikzpicture}

\end{center}
\newpage

\misterio{MORTE DE JESUS}
\begin{center}
  Fruto: A mortificação do espírito.
\end{center}
\sayy{Depois de o haverem crucificado, dividiram suas vestes entre si, tirando a sorte. Cumpriu-se assim a profecia do profeta: Repartiram entre si minhas vestes e sobre meu manto lançaram a sorte \textit{(Sl 21,19)}. Por cima de sua cabeça penduraram um escrito trazendo o motivo de sua crucificação: Este é Jesus, o rei dos judeus. Ao mesmo tempo foram crucificados com ele dois ladrões, um à sua direita e outro à sua esquerda. Próximo da hora nona, Jesus exclamou em voz forte: meu Deus, meu Deus, por que me abandonaste? Jesus de novo lançou um grande brado, e entregou a alma.}

\begin{flushright}
  --- (\textit{cf.} Mt 27:35-50)
\end{flushright}

\begin{center}
  \begin{tikzpicture}[
  scale=0.8,
  every node/.style={anchor=west, align=justify, text width=14cm, font=\fontsize{10}{12}\selectfont},
  ]
  \linha

\conta{Honrar as Cinco Chagas de Jesus e o Sangue que derramou na Cruz}
\conta{Honrar Seu Coração Traspassado e a Cruz em que foi crucificado}
\conta{Honrar a enorme dor que sentiu pelos cravos e a lança que O atravessaram}
\conta{Honrar a vergonha e a infâmia que sofreu, sendo crucificado entre dois ladrões}
\conta{Honrar a compaixão de Sua Mãe Santíssima}
\conta{Honrar Suas sete últimas palavras na Cruz}
\conta{Honrar Seu desamparo e Seu silêncio}
\conta{Honrar a aflição de todo o Universo}
\conta{Honrar Sua morte cruel e ignominiosa}
\conta{Honrar a descida da Cruz e Seu sepultamento}

  \jaculatorias
  \end{tikzpicture}

\end{center}
\newpage

\newpage

