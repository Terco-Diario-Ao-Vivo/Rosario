\section{Mistérios Gozosos}
\misterio{ANUNCIAÇÃO DO ANJO E ENCARNAÇÃO DO VERBO}
\begin{center}
  Fruto: A humildade
\end{center}

\sayy{Entrando, o anjo disse-lhe: \sayy{Ave, cheia de graça, o Senhor é contigo. Não temas, Maria, pois encontraste graça diante de Deus. Eis que conceberás e darás à luz um filho, e lhe porás o nome de Jesus. Ele será grande e será chamado Filho do Altíssimo, e o Senhor Deus lhe dará o trono de seu pai Davi; e reinará eternamente na casa de Jacó, e o seu reino não terá fim}.} 

\begin{flushright}
  --- (\textit{cf.} Lc 1:28-33)
\end{flushright}


\begin{center}
  \begin{tikzpicture}[
    scale=0.8,
    every node/.style={anchor=west, align=justify, text width=14cm, font=\fontsize{10}{12}\selectfont},
  ]

  \linha{pela imensa Caridade de Deus.}

\conta{para lamentar o desgraçado estado de Adão desobediente, sua justa condenação e a de todos os seus filhos}
\conta{para honrar os desejos dos patriarcas e profetas, que pediam a vinda do Messias}
\conta{para honrar os desejos e as preces da Santíssima Virgem, que apressaram a vinda do Messias}
\conta{para honrar a caridade do Pai Eterno, que nos deu Seu divino Filho}
\conta{para honrar o amor do Filho, que se entregou por nós}
\conta{para honrar a embaixada e a saudação do Arcanjo Gabriel}
\conta{para honrar o temor virginal de Maria}
\conta{para honrar a fé e o consentimento da Santíssima Virgem}
\conta{para honrar a criação da alma e a formação do Corpo de Jesus Cristo no seio de Maria, pelo Espírito Santo}
\conta{para honrar a adoração do Verbo Encarnado, pelos anjos, no seio de Maria}

    \jaculatorias
  \end{tikzpicture}
\end{center}
\newpage

\misterio{VISITAÇÃO DE NOSSA SENHORA A SUA PRIMA ISABEL}
\begin{center}
  Fruto: A caridade fraterna
\end{center}

\sayy{Naqueles dias, Maria se levantou e foi às pressas às montanhas, a uma cidade de Judá. Entrou em casa de Zacarias e saudou Isabel. Ora, apenas Isabel ouviu a saudação de Maria, a criança estremeceu no seu seio; e Isabel ficou cheia do Espírito Santo. E exclamou em alta voz: \sayy{Bendita és tu entre as mulheres e bendito é o fruto do teu ventre. Donde me vem esta honra de vir a mim a mãe de meu Senhor? Pois assim que a voz de tua saudação chegou aos meus ouvidos, a criança estremeceu de alegria no meu seio. Bem-aventurada és tu que creste, pois se hão de cumprir as coisas que da parte do Senhor te foram ditas!}.}

\begin{flushright}
  --- (\textit{cf.} Lc 1:39-45)
\end{flushright}

\begin{center}
  \begin{tikzpicture}[
    scale=0.8,
    every node/.style={anchor=west, align=justify, text width=14cm, font=\fontsize{10}{12}\selectfont},
  ]

  \linha{pela adorável Majestade de Deus.}

\conta{para honrar a alegria do Coração de Maria e a morada durante nove meses, do Verbo em seu seio}
\conta{para honrar o sacrifício que Jesus Cristo fez de si mesmo ao Pai, ao entrar neste Mundo}
\conta{para honrar as complacências de Jesus no seio humilde e virginal de Maria, e de Nossa Senhora, na graça de Seu Deus}
\conta{para honrar a humilhação suportada por Maria por causa da dúvida de São José acerca da sua milagrosa maternidade virginal}
\conta{para honrar a eleição dos escolhidos, combinada entre Jesus e Maria, em seu seio}
\conta{para honrar o fervor e caridade de Maria, para anunciar Jesus Cristo, na Sua visita a Santa Isabel}
\conta{para honrar a santificação de João Batista no ventre de sua mãe}
\conta{para honrar a gratidão da Santíssima Virgem com Deus, no Magnificat}
\conta{para honrar a caridade e humildade de Maria em servir sua prima Isabel}
\conta{para honrar a mútua dependência de Jesus e de Maria, e a adoração que deve-mos ter para com Ele e a veneração para com ela}

    \jaculatorias
  \end{tikzpicture}
\end{center}
\newpage

\misterio{NASCIMENTO DE JESUS}
\begin{center}
  Fruto: O espírito de pobreza
\end{center}

\sayy{Maria deu à luz seu filho primogênito. Envolvendo-o em faixas, reclinou-o num presépio, porque não havia lugar para eles na hospedaria. Havia nos arredores uns pastores, que vigiavam e guardavam seu rebanho nos campos durante as vigílias da noite. Um anjo do Senhor apareceu-lhes e a glória do Senhor refulgiu ao redor deles, e tiveram grande temor. O anjo disse-lhes: \sayy{Não temais, eis que vos anuncio uma Boa-Nova que será alegria para todo o povo: hoje vos nasceu na Cidade de Davi um Salvador, que é o Cristo Senhor}.}

\begin{flushright}
  --- (\textit{cf.} Lc 2:7-11)
\end{flushright}

\begin{center}
  \begin{tikzpicture}[
    scale=0.8,
    every node/.style={anchor=west, align=justify, text width=14cm, font=\fontsize{10}{12}\selectfont},
  ]

  \linha{pelas infinitas Riquezas de Deus.}

\conta{para honrar os sofrimentos e humilhações que suportaram, causados pelos desprezos e injúrias feitas a Maria e a São José em Belém}
\conta{para honrar a pobreza do estábulo onde Deus veio ao mundo}
\conta{para honrar a alta contemplação e o excessivo amor de Maria no momento de dar à luz}
\conta{para honrar a saída do Verbo Eterno do seio de Maria mantendo suavirgindade intacta}
\conta{para honrar as adorações e cânticos dos Anjos no nascimento de Jesus}
\conta{para honrar a formosura arrebatadora de Sua divina infância}
\conta{para honrar a vinda dos pastores ao estábulo, com seus presentes}
\conta{para honrar a circuncisão de Jesus Cristo e Suas dores amorosas}
\conta{para honrar a imposição do nome de Jesus Cristo e Suas grandezas}
\conta{para honrar a adoração dos Reis Magos ao menino Jesus Cristo e os presentes que eles Lhe deram}

    \jaculatorias
  \end{tikzpicture}
\end{center}
\newpage

\misterio{PURIFICAÇÃO DE NOSSA SENHORA E APRESENTAÇÃO DE JESUS NO TEMPLO}
\begin{center}
  Fruto: A castidade e a obediência
\end{center}

\sayy{Havia em Jerusalém um homem chamado Simeão. Esse homem, justo e piedoso, esperava a consolação de Israel, e o Espírito Santo estava nele. Impelido pelo Espírito Santo, foi ao templo. E tendo os pais apresentado o menino Jesus, para cumprirem a respeito dele os preceitos da Lei, tomou-o em seus braços e louvou a Deus nestes termos: \sayy{Agora, Senhor, deixai o vosso servo ir em paz, segundo a vossa palavra. Porque os meus olhos viram a vossa salvação que preparastes diante de todos os povos, como luz para iluminar as nações, e para a glória de vosso povo de Israel}.}

\begin{flushright}
  --- (\textit{cf.} Lc 2:25-32)
\end{flushright}

\begin{center}
  \begin{tikzpicture}[
    scale=0.8,
    every node/.style={anchor=west, align=justify, text width=14cm, font=\fontsize{10}{12}\selectfont},
  ]

   \linha{pela eterna Sabedoria de Deus}

\conta{para honrar a obediência de Jesus e de Maria à Lei}
\conta{para honrar o sacrifício que ali fez Jesus de sua Humanidade}
\conta{para honrar o sacrifício que ali fez Maria de Sua honra ao submeter-se à Lei mesmo não precisando ser purificada;}
\conta{para honrar a felicidade e os cânticos de Simeão e Ana, a profetisa}
\conta{para honrar o resgate de Jesus pela oferenda de duas rolinhas}
\conta{para honrar o martírio dos santos inocentes por ordem do rei Herodes}
\conta{para honrar a obediência de São José, à voz do anjo, que guardou Jesus e Maria na sua fuga para o Egito}
\conta{para honrar a estadia misteriosa da Sagrada Família no Egito}
\conta{para honrar a volta da Sagrada Família para o seu lar em Nazaré}
\conta{para honrar o crescimento de Jesus em idade, sabedoria e graça}

    \jaculatorias
  \end{tikzpicture}
\end{center}
\newpage

\misterio{ENCONTRO DE JESUS NO TEMPLO}
\begin{center}
  Fruto: A procura de Deus em todas as coisas
\end{center}

\sayy{O menino ia crescendo e se fortificava: estava cheio de sabedoria, e a graça de Deus repousava nele. Seus pais iam todos os anos a Jerusalém para a festa da Páscoa. Tendo ele atingido doze anos, subiram a Jerusalém, segundo o costume da festa. Acabados os dias da festa, quando voltavam, ficou o menino Jesus em Jerusalém, sem que os seus pais o percebessem. Pensando que ele estivesse com os seus companheiros de comitiva, andaram caminho de um dia e o buscaram entre os parentes e conhecidos. Mas não o encontrando, voltaram a Jerusalém, à procura dele. Três dias depois o acharam no templo, sentado no meio dos doutores, ouvindo-os e interrogando-os. Todos os que o ouviam estavam maravilhados da sabedoria de suas respostas. Quando eles o viram, ficaram admirados. E sua mãe disse-lhe: \sayy{Meu filho, que nos fizeste?! Eis que teu pai e eu andávamos à tua procura, cheios de aflição}. Respondeu-lhes: \sayy{Por que me procuráveis? Não sabíeis que devo ocupar-me das coisas de meu Pai?}.}

\begin{flushright}
  --- (\textit{cf.} Lc 2:40-49)
\end{flushright}

\begin{center}
  \begin{tikzpicture}[
    scale=0.8,
    every node/.style={anchor=west, align=justify, text width=14cm, font=\fontsize{10}{12}\selectfont},
  ]

  \linha{pela Incompreensível Santidade de Deus}

\conta{para honrar Sua vida oculta, laboriosa e obediente na casa de Nazaré}
\conta{para honrar Sua pregação e encontro no Templo entre os doutores}
\conta{para honrar Seu jejum e tentações no deserto}
\conta{para honrar Seu batismo por São João Batista}
\conta{para honrar Sua pregação admirável}
\conta{para honrar Seus milagres portentosos}
\conta{para honrar a eleição de seus doze Apóstolos e os poderes que lhes dá}
\conta{para honrar Sua transfiguração maravilhosa no Monte Tabor}
\conta{para honrar Sua suma humildade no lava-pés dos Apóstolos}
\conta{para honrar a instituição da Sagrada Eucaristia}

    \jaculatorias
  \end{tikzpicture}
\end{center}
\newpage

\newpage

