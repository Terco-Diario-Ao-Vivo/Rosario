\documentclass[a4paper,14pt]{extarticle} \usepackage[utf8]{inputenc}
\usepackage{tikz}
\usepackage{multicol}
\usepackage{fp}
\usepackage[T1]{fontenc}
\usepackage[margin=2.5cm]{geometry}

% Fonte Caladea se existir, senão lmodern
\IfFileExists{caladea.sty}{
  \usepackage{caladea}
}{
  \usepackage{lmodern} }
\usepackage{ragged2e}
\usepackage{wasysym}
\usepackage{graphicx}
\usepackage[portuguese]{babel}
\usepackage{wrapfig}
\usepackage{hyperref}
\usepackage{fancyhdr}
\usepackage{xcolor}
\usepackage{rotating}
\usepackage{titlesec}
\usepackage{epigraph}
\usepackage{dirtytalk}
\usepackage{indentfirst} % Indenta o primeiro parágrafo após seções

\author{
  \href{https://t.me/tercoaovivo}{Canal Terço Diário Ao Vivo}
}

\title{
  O Rosário
}

\date{}
% Ajuste do recuo de parágrafo
\setlength{\parindent}{1.5em}

% Centralizar títulos
\titleformat{\section}
  {\normalfont\centering\bfseries\Large}{\thesection}{1em}{}

\titleformat{\subsection}
  {\normalfont\centering\bfseries\large}{\thesubsection}{1em}{}

\titleformat{\subsubsection}
  {\normalfont\centering\bfseries}{\thesubsubsection}{1em}{}


% -------------- Símbolos de Versículo e Resposta --------------
% Definição do símbolo (a “barrinha” inclinada)
\makeatletter
\newcommand{\vers@resp@sym}{%
  \raisebox{0.2ex}{\rotatebox[origin=c]{-20}{$\m@th\rceil$}}%
}
% macro interna que sobrepõe a barrinha e a letra V ou R
\newcommand{\vers@resp}[2]{%
  {\ooalign{%
     \hidewidth\kern#1\vers@resp@sym\hidewidth\cr
     #2\cr
  }}%
}
% comandos públicos \versicle e \response
\DeclareRobustCommand{\versicle}{\vers@resp{-0.1em}{V}}
\DeclareRobustCommand{\response}{\vers@resp{0pt}{R}}
\makeatother
% ^------------- Símbolos de Versículo e Resposta -------------^

% % Rodapé com imagem e página
% \pagestyle{fancy}
% % ---- Cabeçalho ------------
% \fancyhf[C]{
% \setlength{\headheight}{17.0pt}
% }
% % ----- Rodapé --------------
% \fancyfoot[LO,LE]{%
%   \includegraphics[scale=0.2]{assets/cross.png}\quad
%   \textit{Santo Terço Diário}
% }

\begin{document}

 % variáveis das contas 
\newdimen\ypos
\ypos=-1.5cm

\newcount\numConta
\numConta=1


\newcommand{\conta}[1]{%
  \ifnum\numConta>9
    \def\xshift{-0.475}%
  \else
    \def\xshift{-0.35}%
  \fi

  % círculos e nodes da conta
  \draw[fill=white, thick] (0,\the\ypos) circle (0.3);
  \node at (\xshift,\the\ypos) {\the\numConta};
  \node at (1,\the\ypos) {\textbf{Rezemos esta \nameref{ave-maria}} #1};

  % abaixa a posição y para a próxima conta
  \advance\ypos by -1.7cm
  \advance\numConta by 1

  % reseta o contador quando passar de 10 contas.
  \ifnum\numConta>10
      \setcounter{numConta}{1}
  \fi
}

\newcommand{\linha}[0]{
\draw[gray!70, line width=0.6pt] (0,0) -- (0,-18);
% Conta inicial (Pai Nosso)
\draw[fill=white, thick] (0,0) circle (0.5);
\node at (1,0) {\textbf{Rezemos este \nameref{pai-nosso}} pela essencial Felicidade de Deus...};
}

% última conta
\newcommand{\jaculatorias}[0]{
  \draw[fill=white, thick] (0,\the\ypos) circle (0.5);
  \node at (1,\the\ypos) {\textbf{Finaliza-se com \nameref{oracoes-jaculatorias}}};
}

\vfill
\maketitle
\vfill

\thispagestyle{empty}
\newpage

\tableofcontents


% \par\noindent\rule{\textwidth}{0.4pt}

% \tableofcontents

\newpage
\section{Como utilizar este Documento}

Este documento foi elaborado pensando em ajudar não só os católicos mais experientes, como também os neófitos que nunca nem mesmo rezaram. Contém, em cada mistério do Terço, excertos das sagradas escrituras para contextualizar o leitor e introduzir aos que não os conhecem. 
O documento é repleto de referências "clicáveis", de modo que todas as menções a "\nameref{ave-maria}" (inclusive essa) levam ao texto completo, assim como a maioria das outas orações abreviadas, para que mesmo que o leitor não as conheça, ou até mesmo esqueça, possa seguir a meditação sem dificuldades.

\newpage

\section{Como rezar o Rosário}

\subsection{A oração do Santo Rosário}

Conta-se que a oração do Santo Rosário não surgiu com a estrutura tal qual se encontra nos dias de hoje. Ao longo dos séculos, essa antiga prática de oração foi sofrendo diversas transformações ao mesmo tempo em que ia sendo difundida entre os cristãos católicos de todo o mundo. Cabe recordar que esta singela oração jamais encontrara resistências entre os católicos, ao contrário, sua simplicidade cativante sempre atingiu o coração e a vida dos fiéis que buscavam uma espiritualidade verdadeira, aquela que os apontasse para o céu.

Segundo uma explicação enormemente difundida, a oração do Rosário foi inspirada na vida dos monges. Os fiéis leigos, ao observarem a prática da oração contínua dos 150 salmos que os monges realizavam dentro dos mosteiros e conventos, passaram a querer imitá-los de alguma forma. Como muitos não sabiam ler ou não dispunham de tempo suficiente para cumprir esta prática monacal, em vez de 150 salmos, esses fiéis passaram a rezar 150 \nameref{ave-maria}s.

Posteriormente, as 150 \nameref{ave-maria}s foram divididas em grupos de 50 \nameref{ave-maria}s (terço), que, por sua vez, foram sendo subdivididas em grupos de 10 \nameref{ave-maria}s (cada dezena dos mistérios). Isso explica, inclusive, o nome daquele objeto que comumente é usado nesta prática de oração, ou seja, o “terço”. O terço, portanto, é composto por 50 \nameref{ave-maria}s, quer dizer, a terça parte das 150 \nameref{ave-maria}s que compõem o rosário completo.

Porém, hoje em dia, o rosário não conta mais com 150 \nameref{ave-maria}s, mas com 200. Essa alteração se deu recentemente, no ano de 2002, pelo Papa São João Paulo II. Por meio da carta apostólica “Rosarium Virginis Mariae” (O Rosário da Virgem Maria), o Papa incentivou vividamente o acréscimo de mais um grupo de 50 \nameref{ave-maria}s à oração do Rosário (este novo grupo foi chamado de “mistérios luminosos”). Contudo, o nome da oração das 50 \nameref{ave-maria}s permaneceu sendo chamado de “Terço” ou “Santo Terço”.
\subsection{Os mistérios do Santo Rosário}

Sendo assim, atualmente a oração completa do Rosário da Virgem Maria é composta por quatro Mistérios que relembram os principais episódios da vida de Nosso Senhor Jesus Cristo, são eles: os Mistérios Gozosos (ou mistérios da alegria), os Mistérios Dolorosos (ou mistérios da dor), os Mistérios Gloriosos (ou mistérios da glorificação de Jesus) e, por fim, os Mistérios Luminosos (ou mistérios da luz).

Não existe uma regra ou uma imposição para a oração diária desses mistérios do Santo Rosário. Cada fiel poderá estabelecer, segundo as suas possibilidades, a forma como se encaixa melhor à sua vida. Por exemplo, aqueles que puderem, podem rezar os quatro mistérios do Santo Rosário a cada dia. Quem não possui essa disponibilidade, pode rezar um único mistério (terço) por dia. O mais importante é a fidelidade na oração diária.

Pensando nisso, o Papa São João Paulo II, de maneira muito pedagógica, sugeriu (não impôs) que cada um dos mistérios fosse rezado em dias específicos da semana. Sendo assim, a divisão ficou da seguinte forma: segunda-feira e sábado (Mistérios Gozosos), terça e sexta-feira (Mistérios Dolorosos), quarta-feira e domingo (Mistérios Gloriosos) e quinta-feira (Mistérios Luminosos).

Algo importante deve ser dito sobre o Santo Rosário: ele pode ser rezado particularmente ou em grupo. Além disso, o Rosário é uma forma de oração vocal ou mental, ou seja, pode ser rezado mentalmente (sem a presença da voz) ou em voz audível. Cada pessoa e conforme as circunstâncias discernirá qual modalidade será usada para rezar o Rosário. Porém, São Luís Maria Grignion de Montfort no seu livro “O admirável segredo do Santíssimo Rosário” nos orienta a fazer um esforço para unir a oração vocal e a mental durante a recitação do Santo Rosário.

\subsection{Como rezar o santo terço?}

Agora que já conhecemos um pouco mais sobre o Santo Rosário, chegou o momento de praticarmos. Antes de dar início a oração propriamente dita, pode-se fazer um breve momento de entrega de suas intenções, talvez um pedido, um agradecimento, a apresentação das intenções. Esse momento pode ser bem espontâneo, porém, existem várias orações de oferecimento já compostas, por exemplo:

Divino Jesus, eu vos ofereço este Terço, que vou rezar, contemplando os mistérios de nossa Redenção. Concedei-me, pela intercessão de Maria, vossa Mãe Santíssima, a quem me dirijo, as virtudes que me são necessárias para bem rezá-lo e a graça de ganhar as indulgências anexas a esta santa devoção.

Inicia-se, em seguida, a oração do Rosário com o \textbf{\nameref{sinal-da-cruz}}

Em seguida, segurando a cruz do terço, reza-se o \textbf{\nameref{creio}}.

Na primeira conta do terço (logo depois da cruz), reza-se a oração do \textbf{\nameref{pai-nosso}.}

Passa-se, então, para a 1ª das três contas menores do terço. Em cada uma dessas contas, reza-se uma Ave-Maria dando um total de três \nameref{ave-maria}s. É comum oferecer cada uma dessas \nameref{ave-maria}s às pessoas da Santíssima Trindade, ou seja, uma Ave-Maria para o Pai, outra para o Filho e a última para o Espírito Santo.

Conclui-se esta primeira parte do terço com a oração do \nameref{gloria}.

A partir desse momento, anuncia-se o tema do primeiro Mistério, por exemplo: “Neste primeiro mistério gozoso, contemplo (contemplamos) a anunciação do anjo à Virgem Maria”. (Ao final deste artigo, deixarei a lista completa dos quatro Mistérios do Rosário).

Depois de anunciar o primeiro mistério, reza-se um \nameref{pai-nosso} e dez \nameref{ave-maria}s. Ao concluir as dez \nameref{ave-maria}s, reza-se novamente o “Glória ao Pai” como mencionado acima. Em seguida, reza-se uma oração que a própria Virgem Maria pediu para que rezássemos por ocasião de suas aparições em Fátima, Portugal. Eis a oração:

“Ó meu Jesus, perdoai-nos, livrai-nos do fogo do inferno. Levai as almas todas para o céu e socorrei principalmente as que mais precisarem”.

Pode-se inserir, aqui, uma ou mais jaculatórias da preferência de quem está rezando. (jaculatórias são orações ou invocações bem curtas, por exemplo: “Ó Maria concebida sem pecado, rogai por nós, que recorremos a Vós”, “Santa Maria, rainha dos céus, rogai por nós”, “Jesus, Maria e José, a minha família vossa é” etc.).

Em seguida, anuncia-se o segundo mistério e procede-se como no mistério anterior, rezando um \nameref{pai-nosso}, dez \nameref{ave-maria}s, \nameref{gloria} e jaculatória. Em todos os cinco Mistérios do Terço, procede-se dessa mesma maneira.

Depois de concluir o quinto mistério, o terço já se encaminha para o seu fechamento. Reza-se para concluí-lo a antiquíssima oração da “\nameref{salve-rainha}”, que comumente é precedida por uma breve oração de agradecimento. Então, fica assim:

“Infinitas Graças vos damos, soberana Rainha, pelos benefícios que todos os dias recebemos de vossas mãos liberais. Dignai-vos, agora e para sempre, tomarmos debaixo do vosso poderoso amparo. E para mais vos obrigar, vos saudamos com uma \textbf{\nameref{salve-rainha}}. \textbf{\nameref{salve-rainha}}...”


O Santo Terço, então, é finalizado como começou com o \textbf{\namref{sinal-da-cruz.}}

Caso a pessoa que estiver rezando queira dar seguimento aos demais mistérios na sequência, o agradecimento final e a Salve-Rainha poderão ser rezados somente após o último mistério contemplado, ou seja, a pessoa pode rezar todos os quatro terços (ou os quatro Mistérios) na sequência, não havendo a necessidade de rezar a Salve Rainha ao final de cada terço meditado.

\newpage

\section{Orações}
Esta é uma lista com as orações abreviadas ao longo deste documento, que também são tipicamente usadas no Rosário ou Terço.

\subsection{Credo dos Apóstolos}\label{creio}
Creio em Deus Pai, Todo-poderoso, Criador do Céu e da terra.

E em Jesus Cristo, Seu Único Filho, Nosso Senhor, que foi concebido pelo poder do Espírito Santo; nasceu da Virgem Maria; padeceu sob Pôncio Pilatos; foi crucificado, morto e sepultado; desceu à mansão dos mortos; ressucitou ao terceiro dia; subiu aos céus e está sentado à direita de Deus Pai, Todo-poderoso; donde há de vir a julgar os vivos e os mortos.

Creio no Espírito Santo; na Santa Igreja Católica; na comunhão dos santos; na remissão dos pecados; na ressurreição da carne; na vida eterna. Amém.

\subsection{Sinal da Cruz}\label{sinal-da-cruz}
\footnote{Deve-se levar a mão direita primeiramente à testa, pois foi na testa que, no dia do nosso Batismo, recebemos pela primeira vez o Sinal da Cruz de Cristo. Depois, a mão deve ser dirigida na altura do peito, em seguida, a mão deve ser levada ao ombro esquerdo, depois ao ombro direito. No momento do "amém", não deve ser dado o beijo na mão, pois o próprio "Amém" confirma o que se acabou de rezar.}Em nome do Pai, do Filho e do Espírito Santo. Amém. 

\subsection{Persignação}\label{persignacao}
Pelo sinal da Santa Cruz († cruz na testa)\footnote{Com a mão direita fazer o sinal da cruz, de cima para baixo e da esquerda para a direita… }, livrai-nos Deus, Nosso Senhor († cruz na boca), de nossos inimigos († cruz no peito) .

Amém.

\subsection{Pai-Nosso}\label{pai-nosso}
Pai-Nosso, que estais no céu, santificado seja o Vosso nome; venha a nós o Vosso reino; seja feita a Vossa vontade, assim na terra como no céu. O pão nosso de cada dia nos dai hoje; perdoai-nos as nossas ofensas, assim como nós perdoamos a quem nos tem ofendido; e não nos deixeis cair em tentação, mas livrai-nos do mal. Amém.

\subsection{Ave-Maria}\label{ave-maria}
Ave Maria, cheia de graça, o Senhor é convosco; bendita sois vós entre as mulheres, e bendito é o fruto do vosso ventre, Jesus. Santa Maria, Mãe de Deus, rogai por nós pecadores, agora e na hora da nossa morte. Amém.

\subsection{Salve-Rainha}\label{salve-rainha}
Salve, Rainha, Mãe de misericórdia, vida, doçura e esperança nossa, salve. A vós bradamos, os degredados filhos de Eva, a vós suspiramos, gemendo e chorando neste vale de lágrimas. Eia, pois, advogada nossa, esses vossos olhos misericordiosos a nós volvei, e depois deste desterro, mostrai-nos Jesus, bendito fruto do vosso ventre. Ó clemente, ó piedosa, ó doce sempre Virgem Maria.

\subsection{Glória}\label{gloria}
Glória ao Pai, ao Filho e ao Espírito Santo. Como era no princípio, agora e sempre. Amém.

\subsection{Ó Meu Jesus}\label{oh-meu-jesus}
Ó meu Jesus, perdoai-nos, livrai-nos do fogo do inferno, levai as almas todas para o Céu, principalmente, as que mais precisarem.

\subsection{Jaculatórias}\label{oracoes-jaculatorias}

\subsubsection*{Jaculatória da Imaculada Conceição}\label{jaculatoria-da-imaculada-conceição}
Ó Maria, concebida sem pecado, rogai por nós que recorremos a Vós.

\subsubsection*{Jaculatória do Coração de Jesus.}\label{jaculatoria-do-coracao-de-jesus}

Jesus, manso e humilde de coração, fazei o nosso coração semelhante ao Vosso.

\subsubsection*{Súplica ao Santo de Devoção}\label{suplica-ao-santo-de-devocao}
\textbf{[Nome do santo]}, rogai por nós!

\subsubsection*{Jaculatória do Coração de Jesus.}\label{jaculatoria-do-coracao-de-jesus}
Que a Graça do mistério [citar o mistério] desça em nossas almas.


\subsubsection*{Oração da Virgem em Portugal}\label{oracao-portugal}
Ó meu Jesus, perdoai-nos, livrai-nos do fogo do inferno. Levai as almas todas para o céu e socorrei principalmente as que mais precisarem.

\newpage

\section{Oferecimento} 

Pelo sinal da Santa Cruz [† testa], livrai-nos Deus, Nosso Senhor [† boca], dos nossos inimigos [† coração]. Em nome do Pai, do FIlho e do Espírito Santo. Amém.

Uno-me a todos os Santos que estão no Céu, a todos os justos que estão sobre a Terra, a todas as almas fiéis que estão neste lugar. Uno-me a Vós, meu Jesus, para louvar dignamente Vossa Santa Mãe, e louvar-Vos a vós nela e por ela. Renuncio a todas as distrações que me vierem durante este Terço que quero recitar com modéstia, atenção e devoção como se fosse o último de minha vida. Nós Vos oferecemos, Trindade Santíssima, este Credo, para honrar os mistérios todos de nossa Fé; este Pater e estas três Ave-Marias, para honrar a unidade de Vossa essência e a Trindade de Vossas pessoas. Pedimo-Vos uma fé viva, uma esperança firme e uma caridade ardente. Rezamos também pelas santas almas no purgatório, pelos sacerdotes e religiosos consagrados, pelas vocações e pelo Santo Padre, o papa \textbf{[nome do papa]}. 
Rezamos também pelas seguintes intenções anexas… \textbf{[mencionar as intenções.]}

\vspace{0.5cm}

\textbf{Credo}

Professemos a nossa Fé na presença de Deus; no Evangelho; e como sinal de
obediência ao Papa Leão como Vigário de Jesus Cristo, rezando o Creio.

\vspace{0.5cm}

\textbf{Pai Nosso}

Na Unidade de um só Deus, vivo e verdadeiro rezemos o Pai Nosso.

\vspace{0.5cm}

\textbf{Angelus}

\versicle. O Anjo do Senhor anunciou a Maria

\response. E Ela concebeu pelo Espírito Santo

\textbf{Ave Maria…}

\versicle. Eis a escrava do Senhor.

\response. Faça-se em mim, segundo a Vossa palavra.

\textbf{Ave Maria…}

\versicle. E o Verbo Divino encarnou.

\response. E habitou entre nós.

\textbf{Ave Maria…}

\textbf{Glória ao Pai...}

\newpage

\section{Mistérios Gozosos}
\subsection{ANUNCIAÇÃO DO ANJO E ENCARNAÇÃO DO VERBO}
\begin{center}
  Fruto: A humildade
\end{center}

\say{Entrando, o anjo disse-lhe: \say{Ave, cheia de graça, o Senhor é contigo. Não temas, Maria, pois encontraste graça diante de Deus. Eis que conceberás e darás à luz um filho, e lhe porás o nome de Jesus. Ele será grande e será chamado Filho do Altíssimo, e o Senhor Deus lhe dará o trono de seu pai Davi; e reinará eternamente na casa de Jacó, e o seu reino não terá fim}.}

\begin{center}
  \begin{tikzpicture}[
    scale=0.8,
    every node/.style={anchor=west, align=justify, text width=14cm, font=\fontsize{10}{12}\selectfont},
  ]

    \linha

\conta{Lamentar o desgraçado estado de Adão desobediente, sua justa condenação e a de todos os seus filhos}
\conta{Honrar os desejos dos patriarcas e profetas, que pediam a vinda do Messias}
\conta{Honrar os desejos e as preces da Santíssima Virgem, que apressaram a vinda do Messias}
\conta{Honrar a caridade do Pai Eterno, que nos deu Seu divino Filho}
\conta{Honrar o amor do Filho, que se entregou por nós}
\conta{Honrar a embaixada e a saudação do Arcanjo Gabriel}
\conta{Honrar o temor virginal de Maria}
\conta{Honrar a fé e o consentimento da Santíssima Virgem}
\conta{Honrar a criação da alma e a formação do Corpo de Jesus Cristo no seio de Maria, pelo Espírito Santo}
\conta{Honrar a adoração do Verbo Encarnado, pelos anjos, no seio de Maria}

    \jaculatorias
  \end{tikzpicture}
\end{center}
\newpage

\subsection{VISITAÇÃO DE NOSSA SENHORA A SUA PRIMA ISABEL}
\begin{center}
  Fruto: A caridade fraterna
\end{center}

\say{Naqueles dias, Maria se levantou e foi às pressas às montanhas, a uma cidade de Judá. Entrou em casa de Zacarias e saudou Isabel. Ora, apenas Isabel ouviu a saudação de Maria, a criança estremeceu no seu seio; e Isabel ficou cheia do Espírito Santo. E exclamou em alta voz: \say{Bendita és tu entre as mulheres e bendito é o fruto do teu ventre. Donde me vem esta honra de vir a mim a mãe de meu Senhor? Pois assim que a voz de tua saudação chegou aos meus ouvidos, a criança estremeceu de alegria no meu seio. Bem-aventurada és tu que creste, pois se hão de cumprir as coisas que da parte do Senhor te foram ditas!}.}

\begin{center}
  \begin{tikzpicture}[
    scale=0.8,
    every node/.style={anchor=west, align=justify, text width=14cm, font=\fontsize{10}{12}\selectfont},
  ]

    \linha

\conta{Honrar a alegria do Coração de Maria e a morada durante nove meses, do Verbo em seu seio}
\conta{Honrar o sacrifício que Jesus Cristo fez de si mesmo ao Pai, ao entrar neste Mundo}
\conta{Honrar as complacências de Jesus no seio humilde e virginal de Maria, e de Nossa Senhora, na graça de Seu Deus}
\conta{Honrar a humilhação suportada por Maria por causa da dúvida de São José acerca da sua milagrosa maternidade virginal}
\conta{Honrar a eleição dos escolhidos, combinada entre Jesus e Maria, em seu seio}
\conta{Honrar o fervor e caridade de Maria, para anunciar Jesus Cristo, na Sua visita a Santa Isabel}
\conta{Honrar a santificação de João Batista no ventre de sua mãe}
\conta{Honrar a gratidão da Santíssima Virgem com Deus, no Magnificat}
\conta{Honrar a caridade e humildade de Maria em servir sua prima Isabel}
\conta{Honrar a mútua dependência de Jesus e de Maria, e a adoração que deve-mos ter para com Ele e a veneração para com ela}

    \jaculatorias
  \end{tikzpicture}
\end{center}
\newpage

\subsection{NASCIMENTO DE JESUS}
\begin{center}
  Fruto: O espírito de pobreza
\end{center}

\say{Maria deu à luz seu filho primogênito. Envolvendo-o em faixas, reclinou-o num presépio, porque não havia lugar para eles na hospedaria. Havia nos arredores uns pastores, que vigiavam e guardavam seu rebanho nos campos durante as vigílias da noite. Um anjo do Senhor apareceu-lhes e a glória do Senhor refulgiu ao redor deles, e tiveram grande temor. O anjo disse-lhes: \say{Não temais, eis que vos anuncio uma Boa-Nova que será alegria para todo o povo: hoje vos nasceu na Cidade de Davi um Salvador, que é o Cristo Senhor}.}


\begin{center}
  \begin{tikzpicture}[
    scale=0.8,
    every node/.style={anchor=west, align=justify, text width=14cm, font=\fontsize{10}{12}\selectfont},
  ]

    \linha

\conta{Honrar os sofrimentos e humilhações que suportaram, causados pelos desprezos e injúrias feitas a Maria e a São José em Belém}
\conta{Honrar a pobreza do estábulo onde Deus veio ao mundo}
\conta{Honrar a alta contemplação e o excessivo amor de Maria no momento de dar à luz}
\conta{Honrar a saída do Verbo Eterno do seio de Maria mantendo suavirgindade intacta}
\conta{Honrar as adorações e cânticos dos Anjos no nascimento de Jesus}
\conta{Honrar a formosura arrebatadora de Sua divina infância}
\conta{Honrar a vinda dos pastores ao estábulo, com seus presentes}
\conta{Honrar a circuncisão de Jesus Cristo e Suas dores amorosas}
\conta{Honrar a imposição do nome de Jesus Cristo e Suas grandezas}
\conta{Honrar a adoração dos Reis Magos ao menino Jesus Cristo e os presentes que eles Lhe deram}

    \jaculatorias
  \end{tikzpicture}
\end{center}
\newpage

\subsection{PURIFICAÇÃO DE NOSSA SENHORA E APRESENTAÇÃO DE JESUS NO TEMPLO}
\begin{center}
  Fruto: A castidade e a obediência
\end{center}

\say{Havia em Jerusalém um homem chamado Simeão. Esse homem, justo e piedoso, esperava a consolação de Israel, e o Espírito Santo estava nele. Impelido pelo Espírito Santo, foi ao templo. E tendo os pais apresentado o menino Jesus, para cumprirem a respeito dele os preceitos da Lei, tomou-o em seus braços e louvou a Deus nestes termos: \say{Agora, Senhor, deixai o vosso servo ir em paz, segundo a vossa palavra. Porque os meus olhos viram a vossa salvação que preparastes diante de todos os povos, como luz para iluminar as nações, e para a glória de vosso povo de Israel}.}


\begin{center}
  \begin{tikzpicture}[
    scale=0.8,
    every node/.style={anchor=west, align=justify, text width=14cm, font=\fontsize{10}{12}\selectfont},
  ]

    \linha

\conta{Honrar a obediência de Jesus e de Maria à Lei}
\conta{Honrar o sacrifício que ali fez Jesus de sua Humanidade}
\conta{Honrar o sacrifício que ali fez Maria de Sua honra ao submeter-se à Lei mesmo não precisando ser purificada;}
\conta{Honrar a felicidade e os cânticos de Simeão e Ana, a profetisa}
\conta{Honrar o resgate de Jesus pela oferenda de duas rolinhas}
\conta{Honrar o martírio dos santos inocentes por ordem do rei Herodes}
\conta{Honrar a obediência de São José, à voz doanjo, que guardou Jesus e Maria na sua fuga para o Egito}
\conta{Honrar a estadia misteriosa da Sagrada Família no Egito}
\conta{Honrar a volta da Sagrada Família para o seu lar em Nazaré}
\conta{Honrar o crescimento de Jesus em idade, sabedoria e graça}

    \jaculatorias
  \end{tikzpicture}
\end{center}
\newpage

\subsection{ENCONTRO DE JESUS NO TEMPLO}
\begin{center}
  Fruto: A procura de Deus em todas as coisas
\end{center}

\say{O menino ia crescendo e se fortificava: estava cheio de sabedoria, e a graça de Deus repousava nele. Seus pais iam todos os anos a Jerusalém para a festa da Páscoa. Tendo ele atingido doze anos, subiram a Jerusalém, segundo o costume da festa. Acabados os dias da festa, quando voltavam, ficou o menino Jesus em Jerusalém, sem que os seus pais o percebessem. Pensando que ele estivesse com os seus companheiros de comitiva, andaram caminho de um dia e o buscaram entre os parentes e conhecidos. Mas não o encontrando, voltaram a Jerusalém, à procura dele. Três dias depois o acharam no templo, sentado no meio dos doutores, ouvindo-os e interrogando-os. Todos os que o ouviam estavam maravilhados da sabedoria de suas respostas. Quando eles o viram, ficaram admirados. E sua mãe disse-lhe: \say{Meu filho, que nos fizeste?! Eis que teu pai e eu andávamos à tua procura, cheios de aflição}. Respondeu-lhes: \say{Por que me procuráveis? Não sabíeis que devo ocupar-me das coisas de meu Pai?}.}


\begin{center}
  \begin{tikzpicture}[
    scale=0.8,
    every node/.style={anchor=west, align=justify, text width=14cm, font=\fontsize{10}{12}\selectfont},
  ]

    \linha

\conta{Honrar Sua vida oculta, laboriosa e obediente na casa de Nazaré}
\conta{Honrar Sua pregação e encontro no Templo entre os doutores}
\conta{Honrar Seu jejum e tentações no deserto}
\conta{Honrar Seu batismo por São João Batista}
\conta{Honrar Sua pregação admirável}
\conta{Honrar Seus milagres portentosos}
\conta{Honrar a eleição de seus doze Apóstolos e os poderes que lhes dá}
\conta{Honrar Sua transfiguração maravilhosa no Monte Tabor}
\conta{Honrar Sua suma humildade no lava-pés dos Apóstolos}
\conta{Honrar a instituição da Sagrada Eucaristia}

    \jaculatorias
  \end{tikzpicture}
\end{center}
\newpage

\newpage

\section{Mistérios Dolorosos}
\misterio{AGONIA DE JESUS NO HORTO DAS OLIVEIRAS}
\begin{center}
  Fruto: A detestação dos pecados.
\end{center}

\sayy{Retirou-se Jesus com apóstolos para um lugar chamado Getsêmani e disse-lhes: \say{Assentai-vos aqui, enquanto eu vou ali orar}. E, tomando consigo Pedro e os dois filhos de Zebedeu, começou a se entristecer e a se angustiar. Disse-lhes, então: \say{Minha alma está triste até a morte. Ficai aqui e vigiai comigo}. Adiantou-se um pouco e, prostrando-se com a face por terra, assim rezou: \say{Meu Pai, se é possível, afasta de mim este cálice! Todavia não se faça o que eu quero, mas sim o que tu queres.}}

\begin{flushright}
  --- (\textit{cf.} Mt 26:36-39)
\end{flushright}

\begin{center}
  \begin{tikzpicture}[
  scale=0.8,
  every node/.style={anchor=west, align=justify, text width=14cm, font=\fontsize{10}{12}\selectfont},
]

\linha

\conta{para honrar os divinos retiros que fez Jesus em Sua vida, principalmente no horto...}

\conta{para honrar suas orações humildes e fervorosas durante a vida e na véspera da Paixão...}

\conta{para honrar a paciência e doçura com que suportou Seus Apóstolos, particularmente no Jardim do Getsêmani...}

\conta{para honrar o tédio de Sua Alma durante toda a Sua vida, principalmente no Jardim do Getsêmani...}

\conta{para honrar os rios de sangue que a dor fez brotar de Seu ser adorável...}

\conta{para honrar o consolo que teve por bem aceitar de um Anjo na agonia...}

\conta{para honrar Sua conformidade com a Vontade do Pai, apesar da repugnância ao sofrimento sentida pela Sua natureza humana...}

\conta{para honrar a ignomínia que suportou causada pela traição do Apóstolo Judas Iscariotes e Sua consequente prisão pelos judeus...}

\conta{para honrar a coragem com que saiu ao encontro dos Seus algozes e a força da palavra com que os lançou por terra e os levantou...}

\conta{para honrar a humilhação que suportou e a tristeza que sentiu ao ser abandonado por Seus Apóstolos...}

\jaculatorias
\end{tikzpicture}
\end{center}
\newpage


\misterio{FLAGELAÇÃO DE JESUS}

\begin{center}
  Fruto: A mortificação da carne.
\end{center}

\sayy{Era costume que o governador soltasse um preso a pedido do povo em cada festa de Páscoa. Ora, havia naquela ocasião um prisioneiro famoso, chamado Barrabás. Pilatos dirigiu-se ao povo reunido: Qual quereis que eu vos solte: Barrabás ou Jesus, que se chama Cristo? Mas os príncipes dos sacerdotes e os anciãos persuadiram o povo que pedisse a libertação de Barrabás e fizesse morrer Jesus. O governador tomou então a palavra: Qual dos dois quereis que eu vos solte? Responderam: Barrabás! Pilatos perguntou: Que farei então de Jesus? Todos responderam: Seja crucificado! O governador tornou a perguntar: Mas que mal fez ele? E gritavam ainda mais forte: Seja crucificado! Libertou então Barrabás, mandou açoitar Jesus e lho entregou para ser crucificado.}

\begin{flushright}
  --- (\textit{cf.} Mt 27:15-26)
\end{flushright}

\begin{center}
  \begin{tikzpicture}[
  scale=0.8,
  every node/.style={anchor=west, align=justify, text width=14cm, font=\fontsize{10}{12}\selectfont},
  ]
  \linha

\conta{Honrar o sofrimento que Jesus sentiu causado pelas cordas com que foi atado}
\conta{Honrar a dor e vergonha que Jesus suportou por causa da bofetada que recebeu na casa de Caifás}
\conta{Honrar a tristeza que sentiu por causa das negações de Pedro na casa de Caifás}
\conta{Honrar as ignomínias que aguentou calado na casa de Herodes, quando Lhe puseram veste branca}
\conta{Honrar o opróbrio que suportou no despojamento de Suas vestes}
\conta{Honrar os desprezos e insultos que sofreu de Seus verdugos pela Sua nudez}
\conta{Honrar o sofrimento que aguentou por causa das varas espinhosas e os açoites cruéis com que foi golpeado}
\conta{Honrar a humilhação que Lhe foi causado na coluna em que foi atado}
\conta{Honrar o sangue que derramou e as chagas que recebeu}
\conta{Honrar a Sua queda devido à fraqueza pelo sangue que derramou}

  \jaculatorias
  \end{tikzpicture}

\end{center}
\newpage

\misterio{COROAÇÃO DE ESPINHOS}
\begin{center}
  Fruto: A mortificação do orgulho.
\end{center}

\sayy{Os soldados do governador conduziram Jesus para o pretório e o rodearam com todo o pelotão. Arrancaram-lhe as vestes e lhe colocaram um manto escarlate. Depois, trançaram uma coroa de espinhos, meteram-lha na cabeça e lhe puseram na mão uma vara. Os soldados ajoelharam-se diante dele e diziam com escárnio: Salve, rei dos judeus! Cuspiam-lhe no rosto e, tomando da vara, davam-lhe golpes na cabeça.}

\begin{flushright}
  --- (\textit{cf.} Mt 27:27-31)
\end{flushright}

\begin{center}
  \begin{tikzpicture}[
  scale=0.8,
  every node/.style={anchor=west, align=justify, text width=14cm, font=\fontsize{10}{12}\selectfont},
  ]
  \linha

\conta{Honrar o opróbrio que suportou no despojamento de Suas vestes pela terceira vez}
\conta{Honrar a Sua coroação de espinhos}
\conta{Honrar a mansidão com que aceitou o véu com que Lhe vendaram os olhos}
\conta{Honrar a paciência que aguentou as dores, pelas bofetadas, e as humilhações, pelos escarros com que Lhe cobriram o rosto}
\conta{Honrar a força que teve para suportar o andrajo que Lhe puseram sobre os ombros}
\conta{Honrar ter se deixado pôr uma cana em Suas mãos taumaturgas}
\conta{Honrar a dor que sofreu causada pela pedra pontiaguda sobre a qual O sentaram}
\conta{Honrar a docilidade que demonstrou aos que Lhe desferiram ultrajes e os insultos}
\conta{Honrar o sangue e os suores que saíam de Sua cabeça adorável}
\conta{Honrar a grande dor que suportou ao ter os cabelos e a barba arrancados}

  \jaculatorias
  \end{tikzpicture}

\end{center}
\newpage

\misterio{JESUS CARREGA A CRUZ EM DIREÇÃO AO CALVÁRIO.}
\begin{center}
  Fruto: Paciência nas tribulações.
\end{center}

\sayy{Depois de terem escarnecido dele, lhe tiraram a púrpura, deram-lhe de novo as vestes e o conduziram para fora para que fosse crucificado. Passava por ali certo homem de Cirene, chamado Simão, que vinha do campo e o obrigaram a levar a cruz. Depois conduziram Jesus ao lugar chamado Gólgota, que quer dizer lugar do crânio.}


\begin{flushright}
  --- (\textit{cf.} Mt 27:31-33)
\end{flushright}

\begin{center}
  \begin{tikzpicture}[
  scale=0.8,
  every node/.style={anchor=west, align=justify, text width=14cm, font=\fontsize{10}{12}\selectfont},
  ]
  \linha

\conta{Honrar a apresentação de Nosso Senhor diante do povo com o \sayy{Ecce Homo}}
\conta{Honrar a decepção que sentiu na rejeição que sofreu pelo Seu próprio povo que escolheu Barrabás}
\conta{Honrar como se calou contra os falsos testemunhos que contra Ele deram}
\conta{Honrar a Sua aceitação da condenação à morte}
\conta{Honrar o amor com que abraçou e beijou a Cruz}
\conta{Honrar o trabalho espantoso que teve em carregá-la}
\conta{Honrar a perseverança que continuou o desígnio do Pai apesar das quedas de pura debilidade do corpo sob o peso da Cruz}
\conta{Honrar o encontro doloroso com Sua Santa Mãe}
\conta{Honrar o sudário colocado sobre Sua Santa Face}
\conta{Honrar Suas lágrimas, as de Sua Santa Mãe e das piedosas mulheres que O seguiram até o Monte Calvário}

  \jaculatorias
  \end{tikzpicture}

\end{center}
\newpage

\misterio{MORTE DE JESUS}
\begin{center}
  Fruto: A mortificação do espírito.
\end{center}
\sayy{Depois de o haverem crucificado, dividiram suas vestes entre si, tirando a sorte. Cumpriu-se assim a profecia do profeta: Repartiram entre si minhas vestes e sobre meu manto lançaram a sorte \textit{(Sl 21,19)}. Por cima de sua cabeça penduraram um escrito trazendo o motivo de sua crucificação: Este é Jesus, o rei dos judeus. Ao mesmo tempo foram crucificados com ele dois ladrões, um à sua direita e outro à sua esquerda. Próximo da hora nona, Jesus exclamou em voz forte: meu Deus, meu Deus, por que me abandonaste? Jesus de novo lançou um grande brado, e entregou a alma.}

\begin{flushright}
  --- (\textit{cf.} Mt 27:35-50)
\end{flushright}

\begin{center}
  \begin{tikzpicture}[
  scale=0.8,
  every node/.style={anchor=west, align=justify, text width=14cm, font=\fontsize{10}{12}\selectfont},
  ]
  \linha

\conta{Honrar as Cinco Chagas de Jesus e o Sangue que derramou na Cruz}
\conta{Honrar Seu Coração Traspassado e a Cruz em que foi crucificado}
\conta{Honrar a enorme dor que sentiu pelos cravos e a lança que O atravessaram}
\conta{Honrar a vergonha e a infâmia que sofreu, sendo crucificado entre dois ladrões}
\conta{Honrar a compaixão de Sua Mãe Santíssima}
\conta{Honrar Suas sete últimas palavras na Cruz}
\conta{Honrar Seu desamparo e Seu silêncio}
\conta{Honrar a aflição de todo o Universo}
\conta{Honrar Sua morte cruel e ignominiosa}
\conta{Honrar a descida da Cruz e Seu sepultamento}

  \jaculatorias
  \end{tikzpicture}

\end{center}
\newpage

\newpage


\section{Mistérios Gloriosos}

\misterio{A Ressurreição de Jesus Cristo.}
\begin{center}
  Fruto: A Fé.
\end{center}

\sayy{No primeiro dia da semana, bem de madrugada, as mulheres foram ao túmulo, levando os perfumes que tinham preparado. Encontraram a pedra do tumulo removida, mas, ao entrarem, não encontraram o corpo do Senhor Jesus e ficaram sem saber o que estava acontecendo. Nisso, dois homens com vestes resplandescentes pararam perto delas. Tomadas de medo, elas olhavam para o chão. Eles, porém disseram-lhes: \sayy{Por que procurais entre os mortos aquele que está vivo? Não está aqui. Ressuscitou!}}

\begin{flushright}
  --- (\textit{cf.} Lc 24:1-6)
\end{flushright}

\begin{center}
  \begin{tikzpicture}[
    scale=0.8,
    every node/.style={anchor=west, align=justify, text width=14cm, font=\fontsize{10}{12}\selectfont},
  ]

  \linha{pela Eternidade de Deus, sem princípio.}

\conta{para honrar a descida da Alma de Nosso Senhor aos infernos}
\conta{para honrar o gozo e a saída das almas dos justos que estavam no Limbo}
\conta{para honrar a reunião de Sua Alma e de Seu Corpo no sepulcro}
\conta{para honrar Sua milagrosa saída do Sepulcro}
\conta{para honrar Suas vitórias sobre a morte, o pecado, o mundo e o demônio}
\conta{para honrar os quatro dons gloriosos de Seu Corpo: impassibilidade, agilidade, sutileza e claridade2}
\conta{para honrar o poder que Lhe deu Seu pai no Céu e na Terra}
\conta{para honrar as aparições com que honrou Sua Santa Mãe}
\conta{para honrar as conversações sobre o Céu e a Ceia que fez com Seus Apóstolos}
\conta{para honrar a autoridade e missão que deu a seus discípulos, para que fossem pregar e batizar por toda a Terra}

    \jaculatorias
  \end{tikzpicture}
\end{center}
\newpage

\misterio{A Ascensão de Jesus ao Céu.}
\begin{center}
  Fruto: A esperança e o Desejo do Céu
\end{center}

\sayy{\say{Quem crer e for batizado será salvo, mas quem não crer será condenado. Estes milagres acompanharão os que crerem: expulsarão os demônios em meu nome, falarão novas línguas, manusearão serpentes e, se beberem algum veneno mortal, não lhes fará mal; imporão as mãos aos enfermos e eles ficarão curados.} Depois que o Senhor Jesus lhes falou, foi levado ao céu e está sentado à direita de Deus. Os discípulos partiram e pregaram por toda parte.}

\begin{flushright}
  --- (\textit{cf.} Mc 16:16:20)
\end{flushright}

\begin{center}
  \begin{tikzpicture}[
    scale=0.8,
    every node/.style={anchor=west, align=justify, text width=14cm, font=\fontsize{10}{12}\selectfont},
  ]

  \linha{pela imensidade de Deus, sem limites.}

\conta{para honrar a promessa que fez Jesus aos Apóstolos de Lhes enviar o Espírito Santo, e a ordem que Lhes deu de se prepararem para O receber}
\conta{para honrar a reunião no Monte das Oliveiras}
\conta{para honrar a benção que Lhes deu ao se elevar da Terra aos Céus}
\conta{para honrar Sua gloriosa e admirável Ascensão, por Sua própria virtude, até o Céu Empíreo}
\conta{para honrar o recebimento e o triunfo que Lhe fez Deus, Seu Pai, e toda a Corte Celestial}
\conta{para honrar o poder triunfante com que abriu as portas do Céu, onde nenhum mortal havia entrado}
\conta{para honrar Seu assento à direita do Pai, como Seu Filho querido, igual a Ele mesmo em substância}
\conta{para honrar o poder que Seu Pai Lhe deu de julgar os vivos e os mortos}
\conta{para honrar Sua última vinda sobre a Terra, na qual Seu poder e majestade aparecerão em todo o seu esplendor}
\conta{para honrar a justiça que fará no último Juízo, recompensando os bons e castigando os maus, por toda a eternidade}

    \jaculatorias
  \end{tikzpicture}
\end{center}
\newpage

\misterio{Vinda do Espírito Santo sobre os Apóstolos}
\begin{center}
  Fruto: A Caridade com o próximo
\end{center}

\sayy{Chegando o dia de Pentecostes, estavam reunidos os apóstolos e a Virgem Maria no mesmo lugar. De repente, veio do céu um ruído, como se soprasse um vento impetuoso, e encheu toda a casa onde estavam sentados. Apareceu-lhes então uma espécie de língua de fogo que se repartiu e pousou sobre cada um deles. Ficaram todos cheios do Espírito Santo e começaram a falar em línguas, conforme o Espírito Santo lhes concedia que falassem}.

\begin{flushright}
  --- (\textit{cf. At 2:1-4)}
\end{flushright}
 
\begin{center}
  \begin{tikzpicture}[
    scale=0.8,
    every node/.style={anchor=west, align=justify, text width=14cm, font=\fontsize{10}{12}\selectfont},
  ]

  \linha{pela universal providência de Deus.}

\conta{para honrar a Verdade do Espírito Santo, Deus que procede do Pai e do Filho, e que é o Coração da Divindade}
\conta{para honrar o dom do Espírito Santo pelo Pai e pelo Filho sobre os Apóstolos}
\conta{para honrar o grande estrondo com que o Espírito Santo desceu, sinal de Sua força e poder}
\conta{para honrar as línguas de fogo que o Espírito Santo enviou sobre os Apóstolos, para lhes dar a inteligência das Escrituras, o amor de Deus e do próximo}
\conta{para honrar a plenitude de graças com que o Espírito Santo distinguiu Maria, Sua fiel esposa}
\conta{para honrar a conduta maravilhosa do Espírito Santo, com os santos e com o próprio Jesus Cristo, a quem guiou durante toda a vida}
\conta{para honrar os doze frutos do Espírito Santo: a Caridade, a Alegria, a Paz, a Paciência, a Benignidade, a Bondade, a Longanimidade, a Mansidão, a Fé, a Modéstia, a Continência e a Castidade}
\conta{para honrar os sete dons do Espírito Santo: a sabedoria, a inteligência, o conselho, a fortaleza, a ciência, a piedade e o temor de Deus}
\conta{para pedir, em particular, o dom da Sabedoria e a vinda de Seu Reino aos corações}
\conta{para obter a vitória sobre os três espíritos que Lhe são opostos, a saber: o espírito da carne, do mundo e do demônio}

    \jaculatorias
  \end{tikzpicture}
\end{center}
\newpage

\misterio{Assunção de Maria}
\begin{center}
  Fruto: A Graça de uma boa Morte.
\end{center}

\sayy{Por isto, desde agora, me proclamarão bem-aventurada todas as gerações, porque realizou em mim maravilhas aquele que é poderoso e cujo nome é Santo}.

 
\begin{flushright}
  --- (\textit{cf.} Lc 1:48-49)
\end{flushright}


\begin{center}
  \begin{tikzpicture}[
    scale=0.8,
    every node/.style={anchor=west, align=justify, text width=14cm, font=\fontsize{10}{12}\selectfont},
  ]

  \linha{pela Liberalidade de Deus, inenarrável.}

\conta{para honrar a predestinação eterna de Maria, como obra-prima das mãos de Deus}
\conta{para honrar Sua Conceição Imaculada, a plenitude de graças e o uso da razão no seio de Sua mãe Santa Ana}
\conta{para honrar a Natividade de Maria que regozijou todo o Universo}
\conta{para honrar a apresentação e a vida da infanta Maria no Templo}
\conta{para honrar Maria, pela vida admirável e isenta de todo pecado}
\conta{para honrar a plenitude das virtudes singulares de Maria}
\conta{para honrar Sua virgindade fecunda e Seu parto sem dor}
\conta{para honrar Sua maternidade divina e Sua aliança com a Santíssima Trindade}
\conta{para honrar Sua dormição preciosa e cheia de amor}
\conta{para honrar Sua Ressurreição e Assunção triunfante}

    \jaculatorias
  \end{tikzpicture}
\end{center}
\newpage

\misterio{Coroação de Maria no Céu}
\begin{center}
  Fruto: A devoção filial a Nossa Senhora
\end{center}

\sayy{Apareceu em seguida um grande sinal no céu: uma Mulher revestida do sol, a lua debaixo dos seus pés e na cabeça uma coroa de doze estrelas}.

\begin{flushright}
  --- (\textit{cf.} Ap 12:1)
\end{flushright}


\begin{center}
  \begin{tikzpicture}[
    scale=0.8,
    every node/.style={anchor=west, align=justify, text width=14cm, font=\fontsize{10}{12}\selectfont},
  ]

  \linha{pela glória de Deus, inacessível.}

    \conta{para honrar a tríplice coroa com que a Santíssima Trindade coroou Maria}
    \conta{para honrar o gozo e a glória nova que Maria recebeu o Céu por seu triunfo}
    \conta{para Reconhecer Maria como Rainha do Céu e da Terra, dos Anjos e dos Santos}
    \conta{para honrar a Tesoureira e dispensadora de todas as graças de Deus, dos méritos de Jesus Cristo e dos dons do Espírito Santo}
    \conta{para honrar a Medianeira e Advogada dos homens}
    \conta{para honrar a destruidora e a ruína do demônio e das heresias}
    \conta{para honrar o Seguro Refúgio dos pecadores}
    \conta{para honrar a Mãe e Nutriz dos cristãos}
    \conta{para honrar a que é Alegria e Doçura dos justos}
    \conta{para honrar a que é Asilo Universal dos vivos, Consolo Todo-Poderoso dos aflitos, dos moribundos e das almas do purgatório}

    \jaculatorias
  \end{tikzpicture}
\end{center}
\newpage

\section{Encerramento}
\begin{wrapfigure}{l}{0.28\textwidth}

\begin{center}
  \begin{tikzpicture}[scale=0.8]

    \coordinate (center) at (0,0);
    \coordinate (cross-origin) at (0,-6);

% linha da cruz
\draw[gray!70, line width=0.6pt] (cross-origin) -- (center);

\draw[fill=white] (0,-1.3) circle (0.4) ++ (0,-1) circle (0.25)  ++ (0, -0.875) circle (0.25) ++ (0, -0.875) circle (0.25) ++ (0, -1) circle (0.4);

% contas maiores
% \draw[fill=white] (0,-5) circle (0.4);

% cruz
\draw[black] (cross-origin) -- ++(0.125, 0) -- ++(0,-.25) -- ++(.25,0) -- ++(0,-.25) -- ++ (-.25,0) -- ++(0, -.75) -- ++(-.25,0) -- ++ (0,.75) --++ (-.25,0) -- ++ (0,.25) -- ++ (.25,0) -- ++ (0,.25) -- (cross-origin);

% linha com ângulo
\draw[gray!70, line width=0.6pt, dashed] ({150}:2.7) -- (center);
\draw[gray!70, line width=0.6pt] ({150}:2) -- (center);

\draw[fill=white] ({150}:1.125) circle (0.25);
\draw[fill=white] ({150}:2) circle (0.25);

% linha com ângulo
\draw[gray!70, line width=0.6pt, dashed] ({30}:2.7) -- (center);
\draw[gray!70, line width=0.6pt] ({30}:2) -- (center);

\draw[fill=white] ({30}:1.125) circle (0.25);
\draw[fill=white] ({30}:2) circle (0.25);

% círculo central
\draw[fill=black, thick] (center) circle (0.5);
\draw[gray, <-] (0,.7) --  (0, 1.5);


  \end{tikzpicture}
\end{center}

\end{wrapfigure}

Eu Vos saúdo, Maria, Filha bem-amada do Eterno Pai, Mãe admirável do Filho, Esposa mui fiel do Espírito Santo, templo augusto da Santíssima Trindade; eu Vos saúdo soberana Rainha, a quem tudo está submisso no Céu e na Terra; eu Vos saúdo, seguro refúgio dos pecadores, nossa Senhora da Misericórdia, que jamais repeliste pessoa alguma. Pecador que sou, me prostro aos Vossos pés, e Vos peço de me obter de Jesus, Vosso amado filho, a contrição e o perdão de todos os meus pecados, e a divina sabedoria. Eu me consagro todo a Vós, com tudo o que possuo. Eu Vos tomo, hoje, por minha Mãe e Senhora. Tratai-me, pois, como o último de Vossos filhos e o mais obediente de Vossos escravos. Atendei, minha Rainha, atendei aos suspiros de um coração que deseja amar-Vos e servi-Vos fielmente. Que ninguém diga que, entre todos que a Vós recorreram, seja eu o primeiro desamparado. Ó minha esperança, Ó minha vida, Ó minha fiel e imaculada Virgem Maria defendei-me, nutri-me, escutai-me, instruí-me, guardai-me. Assim seja. \textbf{Salve-Rainha}...

\subsection{Orações reparadoras reveladas pelo Anjo na aparição em Fátima, Portugal,
em 1917:}

Santíssima Trindade, Pai, Filho e Espírito Santo, adoro-Vos profundamente e ofereço-Vos o preciosíssimo Corpo, Sangue, Alma e Divindade de Jesus Cristo, presente em todos os sacrários da Terra, em reparação dos ultrajes, sacrilégios e indiferenças com que Ele mesmo é ofendido. E pelos méritos infinitos do Seu Santíssimo Coração e do Coração Imaculado de Maria, peço-Vos a conversão dos pobres pecadores.

Meu Deus, eu creio, adoro, espero e amo-Vos. Peço-Vos perdão para os que não creem, não adoram, não esperam e não Vos amam. †$^{\ref{sinal-da-cruz}}$

\newpage

\section{Referências e Inspirações}
\begin{itemize}
  \item A formatação deste documento e sua diagramação foi inspirada pelo membro \textbf{\href{https://t.me/phillbush}{Seninha}}, do \textbf{\href{https://t.me/tercoaovivo}{Canal Terço Diário ao Vivo},} que compartilhou um modelo em Troff com ilustrações do terço.
  \item As meditações também extraídas do Método de São Luis Maria Grignion de Montfort, que por sua vez foram retiradas um documento disponibilizado no grupo de \textbf{\href{https://t.me/oracoes_br/2345}{Pedidos de Oração}}.
  \item As passagens anteriores as meditações propostas são extraídas dos terços do aplicativo \textbf{\href{https://pocketterco.com.br/terco/misterios-gloriosos-quarta-e-domingo}{Pocket Terço}}.
  \item As instruções na seção \say{\nameref{como-rezar}} foram adaptadas do texto disponibilizado pela \textbf{\href{https://formacao.cancaonova.com/espiritualidade/oracao/como-rezar-o-santo-rosario/}{Canção Nova}}
\end{itemize}


\end{document}
