\section{Como rezar o Rosário}

\subsection{A oração do Santo Rosário}

Conta-se que a oração do Santo Rosário não surgiu com a estrutura tal qual se encontra nos dias de hoje. Ao longo dos séculos, essa antiga prática de oração foi sofrendo diversas transformações ao mesmo tempo em que ia sendo difundida entre os cristãos católicos de todo o mundo. Cabe recordar que esta singela oração jamais encontrara resistências entre os católicos, ao contrário, sua simplicidade cativante sempre atingiu o coração e a vida dos fiéis que buscavam uma espiritualidade verdadeira, aquela que os apontasse para o céu.

Segundo uma explicação enormemente difundida, a oração do Rosário foi inspirada na vida dos monges. Os fiéis leigos, ao observarem a prática da oração contínua dos 150 salmos que os monges realizavam dentro dos mosteiros e conventos, passaram a querer imitá-los de alguma forma. Como muitos não sabiam ler ou não dispunham de tempo suficiente para cumprir esta prática monacal, em vez de 150 salmos, esses fiéis passaram a rezar 150 \nameref{ave-maria}s.

Posteriormente, as 150 \nameref{ave-maria}s foram divididas em grupos de 50 \nameref{ave-maria}s (terço), que, por sua vez, foram sendo subdivididas em grupos de 10 \nameref{ave-maria}s (cada dezena dos mistérios). Isso explica, inclusive, o nome daquele objeto que comumente é usado nesta prática de oração, ou seja, o “terço”. O terço, portanto, é composto por 50 \nameref{ave-maria}s, quer dizer, a terça parte das 150 \nameref{ave-maria}s que compõem o rosário completo.

Porém, hoje em dia, o rosário não conta mais com 150 \nameref{ave-maria}s, mas com 200. Essa alteração se deu recentemente, no ano de 2002, pelo Papa São João Paulo II. Por meio da carta apostólica “Rosarium Virginis Mariae” (O Rosário da Virgem Maria), o Papa incentivou vividamente o acréscimo de mais um grupo de 50 \nameref{ave-maria}s à oração do Rosário (este novo grupo foi chamado de “mistérios luminosos”). Contudo, o nome da oração das 50 \nameref{ave-maria}s permaneceu sendo chamado de “Terço” ou “Santo Terço”.
\subsection{Os mistérios do Santo Rosário}

Sendo assim, atualmente a oração completa do Rosário da Virgem Maria é composta por quatro Mistérios que relembram os principais episódios da vida de Nosso Senhor Jesus Cristo, são eles: os Mistérios Gozosos (ou mistérios da alegria), os Mistérios Dolorosos (ou mistérios da dor), os Mistérios Gloriosos (ou mistérios da glorificação de Jesus) e, por fim, os Mistérios Luminosos (ou mistérios da luz).

Não existe uma regra ou uma imposição para a oração diária desses mistérios do Santo Rosário. Cada fiel poderá estabelecer, segundo as suas possibilidades, a forma como se encaixa melhor à sua vida. Por exemplo, aqueles que puderem, podem rezar os quatro mistérios do Santo Rosário a cada dia. Quem não possui essa disponibilidade, pode rezar um único mistério (terço) por dia. O mais importante é a fidelidade na oração diária.

Pensando nisso, o Papa São João Paulo II, de maneira muito pedagógica, sugeriu (não impôs) que cada um dos mistérios fosse rezado em dias específicos da semana. Sendo assim, a divisão ficou da seguinte forma: segunda-feira e sábado (Mistérios Gozosos), terça e sexta-feira (Mistérios Dolorosos), quarta-feira e domingo (Mistérios Gloriosos) e quinta-feira (Mistérios Luminosos).

Algo importante deve ser dito sobre o Santo Rosário: ele pode ser rezado particularmente ou em grupo. Além disso, o Rosário é uma forma de oração vocal ou mental, ou seja, pode ser rezado mentalmente (sem a presença da voz) ou em voz audível. Cada pessoa e conforme as circunstâncias discernirá qual modalidade será usada para rezar o Rosário. Porém, São Luís Maria Grignion de Montfort no seu livro “O admirável segredo do Santíssimo Rosário” nos orienta a fazer um esforço para unir a oração vocal e a mental durante a recitação do Santo Rosário.

\subsection{Como rezar o santo terço?}

Agora que já conhecemos um pouco mais sobre o Santo Rosário, chegou o momento de praticarmos. Antes de dar início a oração propriamente dita, pode-se fazer um breve momento de entrega de suas intenções, talvez um pedido, um agradecimento, a apresentação das intenções. Esse momento pode ser bem espontâneo, porém, existem várias orações de oferecimento já compostas, por exemplo:

Divino Jesus, eu vos ofereço este Terço, que vou rezar, contemplando os mistérios de nossa Redenção. Concedei-me, pela intercessão de Maria, vossa Mãe Santíssima, a quem me dirijo, as virtudes que me são necessárias para bem rezá-lo e a graça de ganhar as indulgências anexas a esta santa devoção.

Inicia-se, em seguida, a oração do Rosário com o \textbf{\nameref{sinal-da-cruz}}

Em seguida, segurando a cruz do terço, reza-se o \textbf{\nameref{creio}}.

Na primeira conta do terço (logo depois da cruz), reza-se a oração do \textbf{\nameref{pai-nosso}.}

Passa-se, então, para a 1ª das três contas menores do terço. Em cada uma dessas contas, reza-se uma Ave-Maria dando um total de três \nameref{ave-maria}s. É comum oferecer cada uma dessas \nameref{ave-maria}s às pessoas da Santíssima Trindade, ou seja, uma Ave-Maria para o Pai, outra para o Filho e a última para o Espírito Santo.

Conclui-se esta primeira parte do terço com a oração do \nameref{gloria}.

A partir desse momento, anuncia-se o tema do primeiro Mistério, por exemplo: “Neste primeiro mistério gozoso, contemplo (contemplamos) a anunciação do anjo à Virgem Maria”. (Ao final deste artigo, deixarei a lista completa dos quatro Mistérios do Rosário).

Depois de anunciar o primeiro mistério, reza-se um \nameref{pai-nosso} e dez \nameref{ave-maria}s. Ao concluir as dez \nameref{ave-maria}s, reza-se novamente o “Glória ao Pai” como mencionado acima. Em seguida, reza-se uma oração que a própria Virgem Maria pediu para que rezássemos por ocasião de suas aparições em Fátima, Portugal. Eis a oração:

“Ó meu Jesus, perdoai-nos, livrai-nos do fogo do inferno. Levai as almas todas para o céu e socorrei principalmente as que mais precisarem”.

Pode-se inserir, aqui, uma ou mais jaculatórias da preferência de quem está rezando. (jaculatórias são orações ou invocações bem curtas, por exemplo: “Ó Maria concebida sem pecado, rogai por nós, que recorremos a Vós”, “Santa Maria, rainha dos céus, rogai por nós”, “Jesus, Maria e José, a minha família vossa é” etc.).

Em seguida, anuncia-se o segundo mistério e procede-se como no mistério anterior, rezando um \nameref{pai-nosso}, dez \nameref{ave-maria}s, \nameref{gloria} e jaculatória. Em todos os cinco Mistérios do Terço, procede-se dessa mesma maneira.

Depois de concluir o quinto mistério, o terço já se encaminha para o seu fechamento. Reza-se para concluí-lo a antiquíssima oração da “\nameref{salve-rainha}”, que comumente é precedida por uma breve oração de agradecimento. Então, fica assim:

“Infinitas Graças vos damos, soberana Rainha, pelos benefícios que todos os dias recebemos de vossas mãos liberais. Dignai-vos, agora e para sempre, tomarmos debaixo do vosso poderoso amparo. E para mais vos obrigar, vos saudamos com uma \textbf{\nameref{salve-rainha}}. \textbf{\nameref{salve-rainha}}...”


O Santo Terço, então, é finalizado como começou com o \textbf{\namref{sinal-da-cruz.}}

Caso a pessoa que estiver rezando queira dar seguimento aos demais mistérios na sequência, o agradecimento final e a Salve-Rainha poderão ser rezados somente após o último mistério contemplado, ou seja, a pessoa pode rezar todos os quatro terços (ou os quatro Mistérios) na sequência, não havendo a necessidade de rezar a Salve Rainha ao final de cada terço meditado.

\newpage
